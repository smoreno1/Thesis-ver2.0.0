\chapter{Discussion}
Throughout the design, implementation, and alteration of the imaging pipeline established in this project, several overarching trends emerged which merit further exploration as to how they impact the results of this project and the ability of the pipeline to successfully accomplished the desire goal of being adopted for wider use in biological and medical research. This chapter will serve to delve into these overarching points of discussion, and with it, the personal recommendation and thoughts I had on each subject as they were encountered in the course of  this project.
\medskip

\section{Impact of Scar Tissue On Tissue Clearing, Imaging}
Across all clearing protocols and tissue types (sliced, LV section) used in the course of this project, one recurring pattern noticed was the persistent complications scarred tissue in MI samples introduced when being cleared or imaged using the mesoSPIM system. It had been previously well documented that the scarred, fibrous tissue of myocardium samples tended to be more difficult to completely clear and image compare to healthy cardiac tissue, but the extent to which this was indeed the case was various though never completely absent. 

In CLARITY cleared tissue slices, scarred tissue regions would often not be rendered fully transparent alongside the rest of the healthy tissue regions in the sample. Examples of this can be seen in the following figure.

\begin{figure}[H]
    \centering
        \includegraphics[width=1\linewidth]{Images/Scar_Poor_Clearing.png}
        \caption{\textbf{Reduced quality of CLARITY tissue clearing in scarred tissue (left) compared to healthy (right).} Both slices shown 2.0 mm thick. Poorly cleared scarred tissue region highlighted in red dashed circle. 1 $mm^2$ grid paper background for scale.}
        \label{fig:enter-label}
\end{figure}

While the scarred region was still highly translucent and capable of being imaged, the imperfect clearing still results in may images acquired of scared cardiac regions to be of lower resolution as a result of the increased light scattering compared to healthy, CLARITY cleared cardiac samples of identical thickness. 

This reduction in image resolution from poor clearing resolution in scar regions is less of a factor in tissues cleared using CUBIC-L/RA when compared to CLARITY cleared tissues. The overall lower quality of the clearing achieved by CUBIC based protocol makes the decrease in resolution inside scar tissue regions far less noticeable, with reduction in resolution due to imaging at increasing depths inside the tissue or at increasing distance away from the entry point of the light sheet into the tissue, being far more significant contributors to losses in image resolution and fidelity.

\section{Scheduling and Safety Measures During Implementation of Pipeline}

Through the course of this project, it become very evident that a heavy focus must be made on scheduling the pipeline across the span of multiple months due to the extensive time spans require to clear, image, and process tissues. If a researcher does not properly time and schedule their laboratory session, it can lead to the samples becoming unusable for use in research or requiring researchers to perform time sensitive steps in the pipeline at inconvenient times or dates. 

Based on the experience gained through the implementation of the pipeline with CLARITY and CUBIC-L/RA tissue clearing variations of the pipeline, samples should be processed at the start of the week to ensure no time sensitive steps in the pipeline (such as tissue washing, staining, or solution changes) will need to be performed during weekends when access to laboratory spaces is limited or requires additional permissions to perform these steps using hazardous chemicals out of hours. Conversely, it is recommended that tissue imaging be performed on Thursday or Fridays to allow for multi-day imaging to occur over the weekend when the system is less likely to be disturbed by other lab spacer users. The user will run the risk of the system failing mid-imaging session, potentially causing damage to the inner or outer cuvettes or requiring the entire imaging attempt to be redone from scratch which could set back imaging plans by weeks in the latter case and by months in the former.

To avoid this potential for failure when the system is left alone to complete imaging, it is recommended that any user of the mesoSPIM system run the microscope with the desired imaging settings and tiling parameters assigned with the exception of the z-step setting, which is instead set to a step size over 1000 microns. By running the imaging protocol with these large steps, the user can view the millimetre movement of the cuvette and detection path in the mesoSPIM (that would normally take days to traverse) in a matter of just a few minutes to ensure no collisions or software crashes occur in the process. Performing this safety test run is not required for operation of the mesoSPIM and is merely an redundant safety measure to compliment existing software safeguards to prevent collisions or software crashed mid-imaging, but it is still encouraged to be performed as these safeguards can be deactivated or be insufficient for unique imaging protocols such as the oblique mounted tissue sample protocol described in Chapter 2 and 4.  

\section{Data Requirements for Data Collection, Processing}

In the process of developing a high throughput imaging pipeline, the importance of sufficient data storage space was made very evident. Measures had been taken prior to the start of data acquisition in earnest to ensure storage limitations would not be a limiting factor in the course of experimentation. This included the purchase of a computer system to operate the mesoSPIM with 3.72 TB of internal drive storage as well as the purchase of an external 5 TB external hard drive and a Network Attached Storage (NAS) installed with x 6 3.5 TB Solid State Drives (SSDs). Combined the initial PC setup possessed a storage capacity of 29.72 TB. While this space was seen as sufficient for the first months of data acquisition, it became obvious that this was not the case and greater storage capacity would be required. By the end of the data acquisition period of this project, additional storage was added to the mesoSPIM PC system. This was achieved through the installation of 2 additional 20 TB external drives and the replacement of x 2 3.5 TB SSDs from the NAS with x 2 15 TB HDDs. These additions increased the total storage capacity of the mesoSPIM connected system to approximately 95 TB, of which ~2.5 TB were spread across half a dozen PC units connected to the main mesoSPIM PC via the NAS system. 

The main source of data storage limitations came from the large data size image stack files recorded by the mesoSPIM when tiling protocols are performed, which could be up to 12 TBs in size to record the entire volume of a single LV section sample. In order to reduce the impact these files had on storage, TIFF files were converted to IMS file format using the IMARIS File Converter to compress the data files on average by roughly 30\%. File sets that were stitched together into single image stack would also have their individual TIFF and IMS tile files deleted from the drives and replaced with the finalized stitched images to save additional space. Great care was taken to ensure no duplicates or unnecessary copies of files were left in any data storage drive, with the recycling bins regularly cleared out to save as much storage space as possible.

\section{Chapter Summary}

Chapter 6 has served to describe overarching issues experienced and findings made in the course of developing and performing the cardiac structural imaging pipeline in experimental settings. It is anticipated that with these issues and findings made evident, subsequent applications of the imaging pipeline can avoid many of the unnecessary delays, data processing bottlenecks, scheduling conflicts, and damage to imaging hardware that were encountered in the course of this project. With the knowledge of these findings and the steps taken or suggested to mitigate or avoid their occurrence, the imaging pipeline should be able to achieve the highest degree of image fidelity and data output possible.


