\chapter{Future Outlook and Conclusion}

\section{Thesis Summary and Main Findings}
The objective of this thesis was to create and validate a new volumetric imaging pipeline for the analysis of cardiac tissue structure in cardiovascular research. Two dimensional histology is able to provide great levels of detail and information of the tissue structure across thin transmural sections of the heart (typically > 500 microns thick) \cite{watson_myocardial_2019}. But with an ever growing demand for understanding the three dimensional structure of the myocardium, and the alterations to the structure that occur as a result of MIs, the need for new three-dimensional structural imaging techniques has become increasingly prominent. 

Light Sheet Fluorescence Microscopy provides a basis upon which an avenue for volumetric imaging of cardiac structure can exist. Through the application of a light sheet formed by a Gaussian beam, this microscopy system sees the illumination of an entire section of a sample passing through it, allowing a camera attached to a detection objective to record the sample volume one section at a time \cite{voigt_mesospim_2019}. The images stack of these sections can then be reassembled digitally to allow the three-dimensional structure of the cardiac tissue to be observed and analysed. This method, however, has been limited in its ability to be used in this manner by the small FOVs of most traditional light sheet microscopes and the opacity of cardiac tissue making it impossible for light from the Gaussian beam to penetrate deep into the tissue or for the resulting illumination to exit back out of the tissue towards the detection objective. To overcome these limitations, the application of advanced mesoscale LSFM systems and ASLM techniques alongside the application of tissue clearing and mounting protocols will allow this method of volumetric imaging to remain a viable contender to succeed 2D histology as the cardiac structure image acquisition method of choice.  

The individual steps of the finalized imaging pipeline were presented in Figure 1.1, with each step working towards completing acquisition of volumetric images being discussed in the subsequent three chapters. To begin, in chapter 2, I described the mesoscale LSFM system utilized (the upgraded mesoSPIM version 5) used to image samples up to along with the procedures and programming utilized to obtain and process images acquired by the device. The use of this system permitted thick cardiac tissue sections ($\leq90$ $cm^3$) and portion of cardiac walls to be imaged with the large FOV (~216 $mm^2$) provided by an installed sCMOS camera \cite{voigt_mesospim_2019}. In doing so, the pipeline is able to preserve the three dimensional structural of the tissues that would otherwise be lost if thinly sliced for 2D analysis. Chapter 3 proceeded to describe the methods by which leporine hearts were prepared for imaging, starting from \textit{in vivo} to chemically cleared for use in imaging. The clearing protocols used, CLARITY and CUBIC-L/RA were confirmed experimentally to have succeeded in sufficiently clearing the tissues with the structure of the tissue remaining unaffected by the process in CUBIC-L/RA and having only lateral tissue expansion in CLARITY that could be accounted for in subsequent measurements of structure through the use of the coefficient of expansion. Chapter 3 linked the clearing and imaging protocols together with the discussion of the cleared sample mounting, data post-processing, and mount characterization tests performed as the final step in obtaining the desired volumetric images of cardiac structure. 

To further aid in the argument in favour of the imaging pipeline's use in cardiovascular research, chapter 5 demonstrated the application of the imaging pipeline into ongoing cardiovascular biology experiments. The initial imaging results and data obtained from the imaging pipeline have proven promising with many of the mistakes and errors (improper florescent staining, suture fibres contaminants) being a result of inexperience and oversight  rather than faults inherit to imaging pipeline itself. While the processing of data is still ongoing, the qualitative results obtained so far suggests continued use of the imaging pipeline in this research lab is likely to occur. Findings and issues discovered with the imaging pipeline as it was implemented in these projects was the focus of discussion in Chapter 6, which overviewed each of the overarching issues and the measures taken to resolve them to the best of my ability at the time. While most were managed to reasonable degrees, their continued existence in the pipeline is a strong indicator that the imaging pipeline is still far from perfect and optimizations/refinements are still required, which is the subject of the next section.

\newpage
\section{Future Outlook}
There exists multiple means by which the imaging pipeline as it currently exists can be improved upon, as improvements made to just one of the steps in the imaging pipeline can serve to advance the entire process as a whole. Dozens of new tissue clearing, tissue mounting, and light sheet imaging techniques are published on a regular basis, expanding the possibilities of what can be achieved in their respective fields. This, in turn, can aid in future development of the versatile imaging pipeline, which can be easily adjusted to accommodate the latest advancements in LSM and tissue clearing. While these improvements are rather speculative in nature and may or may not occur in the near future, there do exist a number of technical and logistical improvements to the systems involved that could be implemented immediately. These improvements can help to resolve or reduce a number of the challenges and limitations still present in the pipeline as of writing, further aiding in its capacity as a versatile technique for use in research.

\subsection{Future Developments}
\subsubsection{New mesoSPIM Microscope Upgrades, Versions}
Midway through this time period, the mesoSPIM Initiative, developers of the mesoSPIM system, released the latest version of their namesake microscope system, which was quickly implemented into our existing system to the best of our abilities and resources \cite{vladimirov_benchtop_2024}. This upgrade to the mesoSPIM microscope proved to be an invaluable upgrade to the system which resolved several technical issues that existed in the previous model while also adding new features to enhance image quality and throughput. 

Should further upgrades to the mesoSPIM system be released by the mesoSPIM initiative or by other researchers utilizing the microscope, they could easily increase the capabilities of the mesoSPIM system even further still. There are multiple aspects of the system that could be improved upon, such as the complex alignment process, but one of the more prominent for this line of research would be increasing the volume that can be imaged in the system. Such an advancement would allow for tissue imaging of entire LV sections, if not the entire organ, to be performed. This will further mitigate (if not totally eliminate) the need for structure altering slicing and sectioning in the imaging pipeline.

\subsubsection{New Tissue Clearing Techniques, Variations}
In a similar vein, tissue clearing techniques have also seen several new and modified methods be released in recent years for use across a wide variety of tissue types. One such methods released during the course of research is Tartrazine tissue clearing, which allowed for the reversible clearing of murine tissue in live animal models \cite{ou_achieving_2024} Such advancements suggest tissue clearing could one day allow for the imaging of cardiac samples in vivo, improving the accuracy and reliability of structural data that can be acquired of healthy and diseased compared to existing ex vivo means. Though the exiting imaging pipeline is based on imaging of ex vivo samples, the pipeline could still serve as a basis for new pipelines in which such in vivo imaging can be acquired. As new clearing methods continue to emerge, examining their ability to be implemented into the imaging pipeline, potentially even surpassing the results achieved using CLARITY and CUBIC-L/RA methods, is another avenue of research that merits exploration in any future projects that seek to continue where this pipeline establishing project ends. 

\subsection{Continued Interdisciplinary Collaboration}
This pipeline placed a strong focus on the cross collaboration of multiple research fields to resolve challenges encountered in the course of developing this pipeline. These collaboration efforts proved successful the imaging pipeline developed and performed experimentally in this project could experience widespread adoption in the field of experimental cardiac research and through modification could be made to apply to separate fields in medicine and biology. The pipeline also can serve as a means by which new tissue clearing and staining methods can be tested with, using the staining and clearing techniques currently employed as a control for comparison of imaging results obtained. It is hoped that the success of this project will encourage further interdisciplinary collaboration to occur, which has served as barrier blocking advancements in multiple fields as the insurmountable challenges faced by one filed could prove trivial to another, as was the case with tissue opacity in this project. 

\subsection{Existing Limitations}
\subsubsection{Restoration of Non-utilised mesoSPIM System Features}
Despite efforts to implement as many Benchtop mesoSPIM features into the assembled mesoSPIM, several features remained unused or uninstalled due to continued technical issues, limited lab space, and/or time constraints. These included: dual sided excitation pathways (assembled but deactivated to maximize power output achievable on a single side), selectable magnification objective motorized turret (removed due to technical issues), installation of excitation pathway light filters, installation of new wavelength laser lines, and the restoration USB system control panel (uninstalled due to limited lab table space). All of these features could be activated, upgraded, or reinstalled in future imaging efforts provided sufficient space, funding, and time are made available for each. Should these features and upgrades be added, they could: improve the resolution and contrast of images captured, reduce the time required to record entire tissue volumes, and increase the versatility of the system to allowing for the imaging of even larger samples with a wider variety of fluorescent dyes and immunohistochemistry techniques.

\subsubsection{Removal of Data Storage, Processing Limitations}
Considerable delays in data acquisition were encountered through the course of this project as result of these storage limitations. While increasing the storage capacity did resolve these delays, the risk of filling storage to capacity remained a very real possibility to the very end of the project period. Thus I highly recommend that any future users of the imaging pipeline presented here have access to adequate high volume data storage devices or facilities as this high throughput imaging pipeline is more than capable of generating petabytes (1 PB = 1000 TB) worth of data if utilized on a regular basis to image mesoscale volumes of tissue. Sufficient amounts of CPU RAM (>64 GB) and GPU RAM (>24 GB) would also be recommended for any such system to allow for data transferral and processing to proceed on the same PC network simultaneously with minimal slow down.

\newpage
\section{Conclusion}

The cardiac structural imaging pipeline achieves the volumetric imaging of cardiac tissue structure through the combination of mesoscale light sheet fluorescence microscopy, tissue clearing techniques, sample mounting methods, and data processing protocols. By combining these separate technical steps into a single, well documented and established pipeline, a new source of cardiac structural data is formed which can provide essential information to a wide variety of experiments being conducted looking into various aspects of the cardiovascular system. As shown, this can span from examinations of post MI innervation, comparisons of new and existing models of MIs, and even to the examination of methods currently under development for the restoration of cardiac function to damaged hearts.

Heart disease remains the leading cause of death around the globe, with myocardial infarctions being the leading contributor to the development of malignant arrhythmias as well as both sudden cardiac death and gradual heart failure \cite{golla_heart_2025}. The three dimensional structural alterations that occur in cardiac tissue after an MI remains one of the most poorly understood aspects of the condition to this day due, in no small part, to the limitations of 2D histology that has been the means by which this phenomenon was observed and documented for decades. Gaining a grasp of this intricate and dynamic process though imaging pipelines, such as the one developed and discussed in this thesis, will allow research into mapping the function of the heart to be precisely correlated to alterations in structure, showing how each alteration affects the heart and its long term prognosis. 

The improved understanding into the relation between form and function of the beating heart is instrumental in the development of medicines and surgical/therapeutic treatments that are best suited to minimize the impact of myocardial infarctions and delay, if not outright prevent, the subsequent deterioration of the damaged heart's condition into potential cardiac failure. It is hoped that the cardiac tissue structural imaging pipeline created and optimized to the best of my capability in this thesis will serve as a useful contribution to this ultimate goal of ending the catastrophic consequences of these conditions, be it in the near or distant future.



