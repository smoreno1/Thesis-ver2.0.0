\chapter{Tables and Graphs}
\section{Tables}
\begin{table} [ht]
    \centering
    
    \begin{tabular}{cccc}
            \textbf{Laser Line} & \textbf{Wavelength} & \textbf{Documented Max Power}\\
            LuxX CW Diode Laser 647& 647nm & 140mW\\
            OBIS LS Laser 532 & 532nm & 120mW\\
            LuxX CW Diode Laser 488 & 488nm & 100mW\\
            LuxX CW Diode Laser 405&  405nm & 120mW\\
    \end{tabular}
    \medskip
    \caption{\textbf{Laser Lines Installed in LightHUB Ultra\textsuperscript{\textregistered} Laser Engine}}
    \label{tab:my_label}
\end{table}


\begin{table}
    \centering
    \begin{tabular}{cc}
        \textbf{Procedure Step} & \textbf{Time Duration}\\
            \medskip
         1. PBS Washing & 3 Hours\\
             \medskip
         2. PBS-T Washing & 24 Hours\\
            \medskip
         3*. 1:200 Anti-TH Antibody Wash in PBS-T at $4\celsius$ & 72 Hours\\
            \medskip
         4. 1:100 WGA/AF-488 Conjugate Stain Wash & 48 Hours\\
            \medskip
         5*. 1:200 Goat anti-mouse AF 647 Secondary Antibody Wash in PBS-T & 48 Hours\\
            \medskip
         6. 1:1000 DAPI Nuclear Stain Wash in PBS-T & 6 Hours**\\
            \medskip
         6. PBS-T Wash (3x) & 8 Hours (x3)\\
            \medskip
         7. PFA/PBS Wash & 15 Min\\
            \medskip
         8. PBS Washes (x3) & 5 Min (3x)\\
            \medskip
    \end{tabular}
    \medskip
    \caption{\textbf{Finalized Antibody/WGA Conjugate Dual Staining Protocol For CLARITY Cleared Tissue Slices.} All Steps Performed with Gentle Stirring at Room Temperature Unless Otherwise Stated.*Antibody staining steps skipped if DAPI utilized. **DAPI can be added to last 6 hours of prior washing step, step skipped if antibody staining performed.}
    \label{tab:placeholder}
\end{table}

\begin{table}[H]
    \begin{tabular}{cc}
        \textbf{Procedure Step} & \textbf{Time Duration}\\
            \medskip
         1. PBS Washing & 3 Hours\\
             \medskip
         2. PBS-T Wash (3x) & 8 Hours (x3)\\
            \medskip
         3. WGA Conjugate*, Nuclear** Stains Multiple Stain Wash & 132 Hours\\
            \medskip
         4. Continue Multiple Stain Wash at $4\celsius$ & 12 Hours\\
            \medskip
         5. PBS-T Wash (3x) at $4\celsius$ & 8 Hours (x3)\\
            \medskip
         6. PFA/PBS Wash at $4\celsius$ & 24 Hours\\
            \medskip
         7. PBS Washes (x3) & 2 Hours (3x)\\
            \medskip
         8. Incubation In 50\% CUBIC-RA Aqueous Solution & 12 Hours\\
            \medskip
         9. Incubation In 100\% CUBIC-RA Solution & $\geq 96$ Hours\\
            \medskip
    \end{tabular}
    \medskip
    \caption{\textbf{Finalized DAPI/Nuclear/WGA Conjugate Triple Staining Protocol For CUBIC-L/RA  Cleared LV Sections.} *Compatible WGA Conjugates (1:100): WGA/AF-488, WGA/AF-555, WGA/AF-647. **Compatible Nuclear Stains: SYTOX GREEN (1:2500), PI (3:400), DAPI (1:1000). All Steps Performed with Gentle Stirring at Room Temperature Unless Otherwise Stated. DAPI staining can be omitted to make double staining protocol without change to steps or time duration.}
    \label{tab:placeholder}
\end{table}

\begin{table}[H]
    \centering
    \begin{tabular}{ccc}
         \empty & Left Side of LV Wall & Right Side of LV Wall \\
         \medskip
         \empty & WGA-AF 647/SYTOX Green & WGA-AF 488/ PI\\
         \medskip
        Percutaneous Sham & AE0D P Sham L & AE0D P Sham R\\
         \medskip
        Percutaneous MI & 2265 P MI L & 2265 P MI R\\
         \medskip
        Thoracotomy Sham & 38E2 T Sham L & 38E2 T Sham R\\
         \medskip
        Thoracotomy MI & 37D9 T MI L & 37D9 T MI R\\
         \medskip
    \end{tabular}

    \caption{\textbf{LV Tissue Section Series Imaged Via Imaging Pipeline.} Specific leporine models indicated by unique 4 character alphanumeric code. Code combined with abbreviations as convention for sample naming (P = Percutaneous, T = Thoracotomy, L = Left, R = Right, MI = Scarred Tissue, Sham = Healthy Control Tissue).}
    \label{tab:placeholder}
\end{table}

\begin{table}[H]
    \centering
    \begin{tabular}{ccc}
         Sample Name & Cell Membrane Stain & Cell Nuclei Stain \\
         \medskip
         \empty & WGA Conjugate Dyes & Nucleic Acid Dyes\\
         \medskip
        HiPSC 4.2 & AF-647 & SYTOX Green\\
         \medskip
        H9C2 5.1 & AF-647 & SYTOX Green\\
         \medskip
        H9C2 5.2 & AF-647 & SYTOX Green\\
         \medskip
        H9C2 5.3 & AF-555 & SYTOX Green\\
    \end{tabular}
    
    \caption{\textbf{Spheroid injected sample tissue staining, naming conventions.}}
    \label{tab:placeholder}
\end{table}





\section{Graphs}
\begin{figure}[H]
    \centering
    \begin{tikzpicture}
        \begin{axis}
       
        [
        scale = 1,
        ytick={1,2,3},
        yticklabels={Overall,Healthy, Scar},
        boxplot/box extend=0.25,
        ]

\addplot+ [
        boxplot prepared={draw position=1,
          median=1.97,
          upper quartile=2.72,
          lower quartile=1.71,
          upper whisker=3.42,
          lower whisker=1.43},
        ] table [y index=0] {Data/Figure3.3.dat}
        [above]
        node at
          (boxplot whisker cs:\boxplotvalue{lower whisker},1)
          {\pgfmathprintnumber{\boxplotvalue{lower whisker}}}
        node at
          (boxplot box cs: \boxplotvalue{median},1)
          {\pgfmathprintnumber{\boxplotvalue{median}}}
        node at
          (boxplot whisker cs:\boxplotvalue{upper whisker},1)
          {\pgfmathprintnumber{\boxplotvalue{upper whisker}}}
        ;
        
\addplot+[
        boxplot prepared={ draw position=2,
          median=1.98,
          upper quartile=2.03,
          lower quartile=1.67,
          upper whisker=2.03,
          lower whisker=1.67},
        ] table [y index=1] {Data/Figure3.3.dat}
        [above]
        node at
          (boxplot whisker cs:\boxplotvalue{lower whisker},1.2)
          {\pgfmathprintnumber{\boxplotvalue{lower whisker}}}
        node at
          (boxplot box cs: \boxplotvalue{median},-2)
          {\pgfmathprintnumber{\boxplotvalue{median}}}
        node at
          (boxplot whisker cs:\boxplotvalue{upper whisker},1.2)
          {\pgfmathprintnumber{\boxplotvalue{upper whisker}}}
        ;
\addplot+[
        boxplot prepared={draw position=3,
          median=1.96,
          upper quartile=3.18,
          lower quartile=1.62,
          upper whisker=3.42,
          lower whisker=1.43}
        ] table [y index=2] {Data/Figure3.3.dat}
        [above]
        node at
          (boxplot whisker cs:\boxplotvalue{lower whisker},1)
          {\pgfmathprintnumber{\boxplotvalue{lower whisker}}}
        node at
          (boxplot box cs: \boxplotvalue{median},1)
          {\pgfmathprintnumber{\boxplotvalue{median}}}
        node at
          (boxplot whisker cs:\boxplotvalue{upper whisker},1)
          {\pgfmathprintnumber{\boxplotvalue{upper whisker}}}
        ;
        
        \end{axis}
    \end{tikzpicture}
    \caption{\textbf{Surface Area Expansion Coefficient of 500um CLARITY Cleared Tissue Slices.} Median, Upper and Lower Quartiles, Max and Min Extrema Shown (N,overall = 8, n,scar = 5, n,healthy = 3)}
    \label{fig:enter-label}
\end{figure}

\begin{figure}[H]
    \centering
    \begin{tikzpicture}
        \begin{axis}
        [
        scale = 1,
        ytick={1,2,3},
        yticklabels={Overall,Healthy, Scar},
        boxplot/box extend=0.25,
        ]

\addplot+[boxplot] 
            table [y index=0] {Data/CUBIC_Overall.tex}
            [above]
            node at
              (boxplot whisker cs:\boxplotvalue{lower whisker},1)
              {\pgfmathprintnumber{\boxplotvalue{lower whisker}}}
            node at
              (boxplot box cs: \boxplotvalue{median},-1)
              {\pgfmathprintnumber{\boxplotvalue{median}}}
            node at
              (boxplot whisker cs:\boxplotvalue{upper whisker},1)
              {\pgfmathprintnumber{\boxplotvalue{upper whisker}}}
            ;
        
        
\addplot+[boxplot, mark=none] 
            table [y index=0] {Data/CUBIC_Healthy.tex}
            [above]
            node at
              (boxplot whisker cs:\boxplotvalue{lower whisker},1)
              {\pgfmathprintnumber{\boxplotvalue{lower whisker}}}
            node at
              (boxplot box cs: \boxplotvalue{median},-1)
              {\pgfmathprintnumber{\boxplotvalue{median}}}
            node at
              (boxplot whisker cs:\boxplotvalue{upper whisker},1)
              {\pgfmathprintnumber{\boxplotvalue{upper whisker}}}
            ;
        
\addplot+[boxplot, mark=none] 
            table [y index=0] {Data/CUBIC_Scar.tex}
            [above]
            node at
              (boxplot whisker cs:\boxplotvalue{lower whisker},1)
              {\pgfmathprintnumber{\boxplotvalue{lower whisker}}}
            node at
              (boxplot box cs: \boxplotvalue{median},1)
              {\pgfmathprintnumber{\boxplotvalue{median}}}
            node at
              (boxplot whisker cs:\boxplotvalue{upper whisker},1)
              {\pgfmathprintnumber{\boxplotvalue{upper whisker}}}
            ;
        
        
        \end{axis}
    \end{tikzpicture}
    \caption{\textbf{Axial Width Expansion Coefficient of CUBIC Cleared LV Sections.} Median, Upper and Lower Quartiles, Max and Min Extrema Shown (N, overall = 8; n, scar = 4, n; healthy = 4). Data outliers shown as individual data points}
    \label{fig:enter-label}
\end{figure}


\begin{figure}[H]
    \centering
    \begin{subfigure}[t]{0.9\textwidth}
    \centering
    \begin{tikzpicture}
    
    \begin{axis}[
            ymajorgrids,
            xmajorgrids,
            scale= 1.2,
            width = 12cm,
            height = 7cm,
            ylabel={Transmission (\%)},
            xlabel={Wavelength (nm)},
            xmin = 390, xmax = 660,
            ymin = -10, ymax = 105,
            legend style={at={(0.7,0.2)}, 
            anchor=north}
            ]
        \addplot+[
            const plot,
            color=blue,
            mark=square,
            mark size = 3pt,
            error bars/.cd,
                x dir=both, x fixed = 1,
                y dir=both, y fixed = 4.5
            ]
            table [ x index=0, y index=3]{Data/400um_UNCLEARED_2.dat};%400um_CLARITY_UNCLEARED_TRANSMISSION.dat};
        \addplot+[
            color=red,
            mark=o,
            mark size = 3pt,
            error bars/.cd,
                x dir=both, x fixed = 1,
                y dir=both, y fixed = 4.5
            ]
            table [x index=0, y index=2] {Data/400um_CLEARED_2.dat};%CLARITY_Transmission.dat};
        \legend{0.4 mm Uncleared, 0.4mm Cleared}
        \end{axis}
    \end{tikzpicture}
    \caption{\textbf{Percent Light Transmission of Mounted 0.4mm CLARITY Sliced Samples.} Light Transmittance Normalized to Control 400um Mount Filled with EI Solution.}    
    \label{fig:enter-label}
    \end{subfigure}
    \medskip
    
    \begin{subfigure}[t]{0.9\textwidth}
    \centering
    \begin{tikzpicture}
    \begin{axis}[
            ymajorgrids,
            xmajorgrids,
            scale= 1.2,
            width = 12cm,
            height = 7cm,
            xlabel={Wavelength (nm)},
            ylabel={Transmission (\%)},
            xmin = 390, xmax = 660,
            ymin = -10, ymax = 105,
            legend style={at={(0.725,0.2)}, 
            anchor=north}
            ]
        \addplot+[
            const plot,
            color=blue,
            mark=square,
            mark size = 3pt,
            error bars/.cd,
                x dir=both, x fixed = 1,
                y dir=both, y fixed = 4.5
            ]
            table [ x index=0, y index=3] {Data/2mm_CLARITY_UNCLEARED_TRANSMISSION.dat};
        \addplot+[
            color=red,
            mark=o,
            mark size = 3pt,
            error bars/.cd,
                x dir=both, x fixed = 1,
                y dir=both, y fixed = 4.5
            ]
            table [x index=0, y index=2] {Data/2mm_CLARITY_TRANSMISSION.dat};
        \legend{2 mm Uncleared, 2 mm Cleared}
        \end{axis}
    \end{tikzpicture}
    \caption{\textbf{Percent Light Transmission of Mounted 2.0mm CLARITY Sliced Samples.} Light Transmittance Normalized to Control 2mm Mount Filled with EI Solution.}    
    \label{fig:enter-label}
    \end{subfigure}
    \medskip
    
    \caption{\textbf{Comparison of Percent Light Transmission of Mounted Cleared and Uncleared Sliced Cardiac Tissue Samples.}  HORIBA Duetta \% Transmittance Accuracy: $\pm$ 4.5\%, Wavelength Emission Accuracy: $\pm$ 1nm []. }    
    \label{fig:enter-label}
\end{figure}

\begin{figure}[H]
    \centering
    \begin{tikzpicture}
    \begin{axis}[
            ymajorgrids,
            xmajorgrids,
            xmin = 395, xmax = 655,
            ymin = 85, ymax = 100.25,
            scale= 1.2,
            width = 12cm,
            height = 7cm,
            ylabel={Transmission (\%)},
            xlabel={Wavelength (nm)},
            legend style={at={(0.8,0.2)}, 
            anchor=north}
            ]
        
        \addplot+[
            color=orange,
            mark=diamond,
            mark size = 5pt,
            error bars/.cd,
                x dir=both, x fixed = 1,
                y dir=both, y fixed = 4.5
            ]
            table [x index=0, y index=2] {Data/400um_CLARITY_Transmission.dat};
        %\addplot+[
            %color=green,
            %mark=square,
            %mark size = 3pt,
            %error bars/.cd,
               %x dir=both, x fixed = 1,
               %y dir=both, y fixed = 4.5
            %]
            %table [x index=0, y index=1]{Data/1mm_Cleared_2.dat};
        \addplot+[
            color=blue,
            mark=x,
            mark size = 5pt,
            error bars/.cd,
                x dir=both, x fixed = 1,
                y dir=both, y fixed = 4.5
            ]
            table [x index=0, y index=2] {Data/2mm_CLARITY_Transmission.dat};
            \legend{0.4mm, 2.0mm}
        \end{axis}
    \end{tikzpicture}
    \caption{\textbf{Percent Light Transmission of Mounted 0.4 and 2.0 mm CLARITY Cleared, Unstained Sliced Samples.}Tissue Sample Light Transmittance Normalized to Respective Thickness Control Mounts Filled with EI Solution. HORIBA Duetta \% Transmittance Accuracy: $\pm$ 4.5\%, Wavelength Emission Accuracy: $\pm$ 1nm [].}
    \label{fig:enter-label}
\end{figure}


\begin{figure}[H]
    \centering
    \begin{tikzpicture}
    \begin{axis}[
            ymajorgrids,
            xmajorgrids,
            scale= 1.2,
            width = 12cm,
            height = 7cm,
            xmin = 390, xmax = 660,
            ymin = 92.5, ymax = 100.25,
            ylabel={Transmission (\%)},
            xlabel={Wavelength (nm)},
            legend style={at={(0.8,0.5)}, 
            anchor=north}
            ]
        
        \addplot+[
            color=red,
            mark=o,
            mark size = 3pt,
            %error bars/.cd,
                %x dir=both, x fixed = 1,
                %y dir=both, y fixed = 4.5
            ]
            table [x index=0, y index=1] {Data/0.5mmBlankData.dat};
        \addplot+[
            color=green,
            mark=square,
            mark size = 3pt,
            %error bars/.cd,
                %x dir=both, x fixed = 1,
                %y dir=both, y fixed = 4.5
            ]
            table [x index=0, y index=1] 
            {Data/1mmBlankData.dat};
        \addplot+[
            color=blue,
            mark=x,
            mark size = 5pt,
            %error bars/.cd,
                %x dir=both, x fixed = 1,
                %y dir=both, y fixed = 4.5
            ]
            table [x index=0, y index=1] 
            {Data/2mmBlankData.dat};
            \legend{0.5mm, 1.0mm, 2.0mm}
        \end{axis}

        
    \end{tikzpicture}
    \caption{\textbf{Percent Light Transmission of Mounted Control Samples Filled with EI Solution.} Control Light Transmittance Normalized to Control 0.4mm Mount Filled with EI Solution.HORIBA Duetta \% Transmittance Accuracy: $\pm$ 4.5\%, Wavelength Emission Accuracy: $\pm$ 1nm [].}
    \label{fig:enter-label}
\end{figure}

[FIGURE 3.10 HERE]

[FIGURE 4.8 HERE]

\begin{figure}[H]
\centering
    \begin{subfigure}[t]{0.49\textwidth}
        \centering
        \begin{tikzpicture}
            \begin{axis}
            [
            scale = 1,
            xmin = 2,
            xmax = 13,
            ytick={1,2,3},
            yticklabels={Z',Y, X'}, 
            boxplot/box extend=0.25,
            ]
    
    \addplot+ [boxplot] 
            table [y index=0] {Data/B4_2/2zfwhm.txt}
            [above]
            node at
              (boxplot whisker cs:\boxplotvalue{lower whisker},1)
              {\pgfmathprintnumber{\boxplotvalue{lower whisker}}}
            node at
              (boxplot box cs: \boxplotvalue{median},1)
              {\pgfmathprintnumber{\boxplotvalue{median}}}
            node at
              (boxplot whisker cs:\boxplotvalue{upper whisker},1)
              {\pgfmathprintnumber{\boxplotvalue{upper whisker}}}
            ;
            
    \addplot+[boxplot] 
            table [y index=0] {Data/B4_2/2yfwhm.txt}
            [above]
            node at
              (boxplot whisker cs:\boxplotvalue{lower whisker},1)
              {\pgfmathprintnumber{\boxplotvalue{lower whisker}}}
            node at
              (boxplot box cs: \boxplotvalue{median},-1)
              {\pgfmathprintnumber{\boxplotvalue{median}}}
            node at
              (boxplot whisker cs:\boxplotvalue{upper whisker},1)
              {\pgfmathprintnumber{\boxplotvalue{upper whisker}}}
            ;
            
    \addplot+[boxplot] 
            table [y index=0] {Data/B4_2/2xfwhm.txt}
            [above]
            node at
              (boxplot whisker cs:\boxplotvalue{lower whisker},1)
              {\pgfmathprintnumber{\boxplotvalue{lower whisker}}}
            node at
              (boxplot box cs: \boxplotvalue{median},-1)
              {\pgfmathprintnumber{\boxplotvalue{median}}}
            node at
              (boxplot whisker cs:\boxplotvalue{upper whisker},1)
              {\pgfmathprintnumber{\boxplotvalue{upper whisker}}}
            ;
            [
                       
            \end{axis}
        \end{tikzpicture}
        \caption{\textbf{Agarose Bead Mount \#1 (n=xxx)}}
    \end{subfigure}
    \medskip
    ~
    \begin{subfigure}[t]{0.49\textwidth}
        \centering
        \begin{tikzpicture}
             \begin{axis}
            [
            scale = 1,
            xmin = 2,
            xmax = 13,
            ytick={1,2,3},
            yticklabels={Z',Y, X'},
            boxplot/box extend=0.25,
            ]
    
    \addplot+ [boxplot] 
            table [y index=0] {Data/B4_3/3zfwhm.txt}
            [above]
            node at
              (boxplot whisker cs:\boxplotvalue{lower whisker},1)
              {\pgfmathprintnumber{\boxplotvalue{lower whisker}}}
            node at
              (boxplot box cs: \boxplotvalue{median},-1)
              {\pgfmathprintnumber{\boxplotvalue{median}}}
            node at
              (boxplot whisker cs:\boxplotvalue{upper whisker},1)
              {\pgfmathprintnumber{\boxplotvalue{upper whisker}}}
            ;
            
    \addplot+[boxplot] 
            table [y index=0] {Data/B4_3/3yfwhm.txt}
            [above]
            node at
              (boxplot whisker cs:\boxplotvalue{lower whisker},1)
              {\pgfmathprintnumber{\boxplotvalue{lower whisker}}}
            node at
              (boxplot box cs: \boxplotvalue{median},1)
              {\pgfmathprintnumber{\boxplotvalue{median}}}
            node at
              (boxplot whisker cs:\boxplotvalue{upper whisker},1)
              {\pgfmathprintnumber{\boxplotvalue{upper whisker}}}
            ;
            
    \addplot+[boxplot] 
            table [y index=0] {Data/B4_3/3xfwhm.txt}
            [above]
            node at
              (boxplot whisker cs:\boxplotvalue{lower whisker},1)
              {\pgfmathprintnumber{\boxplotvalue{lower whisker}}}
            node at
              (boxplot box cs: \boxplotvalue{median},-1)
              {\pgfmathprintnumber{\boxplotvalue{median}}}
            node at
              (boxplot whisker cs:\boxplotvalue{upper whisker},1)
              {\pgfmathprintnumber{\boxplotvalue{upper whisker}}}
            ;
            [
                       
            \end{axis}
        \end{tikzpicture}
        \caption{\textbf{Agarose Bead Mount \#2 (n=xxx)}}
    \end{subfigure}
    \medskip
    
\begin{subfigure}[t]{0.49\textwidth}
        \centering
        \begin{tikzpicture}
            \begin{axis}
            [
            scale = 1,
            xmin = 2,
            xmax = 13,
            ytick={1,2,3},
            yticklabels={Z',Y, X'},
            boxplot/box extend=0.25,
            ]
    
    \addplot+ [boxplot] 
            table [y index=0] {Data/B4_4/4zfwhm.txt}
            [above]
            node at
              (boxplot whisker cs:\boxplotvalue{lower whisker},1)
              {\pgfmathprintnumber{\boxplotvalue{lower whisker}}}
            node at
              (boxplot box cs: \boxplotvalue{median},1)
              {\pgfmathprintnumber{\boxplotvalue{median}}}
            node at
              (boxplot whisker cs:\boxplotvalue{upper whisker},1)
              {\pgfmathprintnumber{\boxplotvalue{upper whisker}}}
            ;
            
    \addplot+[boxplot] 
            table [y index=0] {Data/B4_4/4yfwhm.txt}
            [above]
            node at
              (boxplot whisker cs:\boxplotvalue{lower whisker},1)
              {\pgfmathprintnumber{\boxplotvalue{lower whisker}}}
            node at
              (boxplot box cs: \boxplotvalue{median},-1)
              {\pgfmathprintnumber{\boxplotvalue{median}}}
            node at
              (boxplot whisker cs:\boxplotvalue{upper whisker},1)
              {\pgfmathprintnumber{\boxplotvalue{upper whisker}}}
            ;
            
    \addplot+[boxplot] 
            table [y index=0] {Data/B4_4/4xfwhm.txt}
            [above]
            node at
              (boxplot whisker cs:\boxplotvalue{lower whisker},1)
              {\pgfmathprintnumber{\boxplotvalue{lower whisker}}}
            node at
              (boxplot box cs: \boxplotvalue{median},-1)
              {\pgfmathprintnumber{\boxplotvalue{median}}}
            node at
              (boxplot whisker cs:\boxplotvalue{upper whisker},1)
              {\pgfmathprintnumber{\boxplotvalue{upper whisker}}}
            ;
            [
                       
            \end{axis}
        \end{tikzpicture}
        \caption{\textbf{Agarose Bead Mount \#3 (n=xxx)}}
    \end{subfigure}
    \medskip
    ~
    \begin{subfigure}[t]{0.49\textwidth}
        \centering
        \begin{tikzpicture}
             \begin{axis}
             [
            scale = 1,
            xmin = 2,
            xmax = 13,
            ytick={1,2,3},
            yticklabels={Z',Y, X'},
            boxplot/box extend=0.25,
            ]
    
    \addplot+ [boxplot] 
            table [y index=0] {Data/B4_5/5zfwhm.txt}
            [above]
            node at
              (boxplot whisker cs:\boxplotvalue{lower whisker},1)
              {\pgfmathprintnumber{\boxplotvalue{lower whisker}}}
            node at
              (boxplot box cs: \boxplotvalue{median},-1)
              {\pgfmathprintnumber{\boxplotvalue{median}}}
            node at
              (boxplot whisker cs:\boxplotvalue{upper whisker},1)
              {\pgfmathprintnumber{\boxplotvalue{upper whisker}}}
            ;
            
    \addplot+[boxplot] 
            table [y index=0] {Data/B4_5/5yfwhm.txt}
            [above]
            node at
              (boxplot whisker cs:\boxplotvalue{lower whisker},1)
              {\pgfmathprintnumber{\boxplotvalue{lower whisker}}}
            node at
              (boxplot box cs: \boxplotvalue{median},1)
              {\pgfmathprintnumber{\boxplotvalue{median}}}
            node at
              (boxplot whisker cs:\boxplotvalue{upper whisker},1)
              {\pgfmathprintnumber{\boxplotvalue{upper whisker}}}
            ;
            
    \addplot+[boxplot] 
            table [y index=0] {Data/B4_5/5xfwhm.txt}
            [above]
            node at
              (boxplot whisker cs:\boxplotvalue{lower whisker},1)
              {\pgfmathprintnumber{\boxplotvalue{lower whisker}}}
            node at
              (boxplot box cs: \boxplotvalue{median},-1)
              {\pgfmathprintnumber{\boxplotvalue{median}}}
            node at
              (boxplot whisker cs:\boxplotvalue{upper whisker},1)
              {\pgfmathprintnumber{\boxplotvalue{upper whisker}}}
            ;
            [
                       
            \end{axis}
        \end{tikzpicture}
        \caption{\textbf{Agarose Bead Mount \#4 (n=xxx)}}
    \end{subfigure}
    \caption{\textbf{Agarose Bead Samples Point Spread Analysis: 4 Custom 3D Printed Mounts, No Shear Technique} Box plots of full width at half maximum (FWHM) results in x',y, and z' sample coordinate axes. Median, Upper and Lower Quartiles, Max and Min Extrema Shown (N,overall = XXX Beads)}
    \label{fig:enter-label}
\end{figure}

\begin{figure}[H] 
    \centering
    \begin{subfigure}[t]{0.49\textwidth}
    \centering
        \begin{tikzpicture}
            \begin{axis}
             [
            scale = 1,
            xmin = 2,
            xmax = 15.5,
            ytick={1,2,3},
            yticklabels={Z',Y, X'},
            boxplot/box extend=0.25,
            ]

            \addplot+ [boxplot] 
                    table [y index=0] {Data/double/zfwhm.txt}
                    [above]
                    node at
                      (boxplot whisker cs:\boxplotvalue{lower whisker},1)
                      {\pgfmathprintnumber{\boxplotvalue{lower whisker}}}
                    node at
                      (boxplot box cs: \boxplotvalue{median},-1)
                      {\pgfmathprintnumber{\boxplotvalue{median}}}
                    node at
                      (boxplot whisker cs:\boxplotvalue{upper whisker},1)
                      {\pgfmathprintnumber{\boxplotvalue{upper whisker}}}
                    ;
                    
            \addplot+[boxplot] 
                    table [y index=0] {Data/double/yfwhm.txt}
                    [above]
                    node at
                      (boxplot whisker cs:\boxplotvalue{lower whisker},1)
                      {\pgfmathprintnumber{\boxplotvalue{lower whisker}}}
                    node at
                      (boxplot box cs: \boxplotvalue{median},1)
                      {\pgfmathprintnumber{\boxplotvalue{median}}}
                    node at
                      (boxplot whisker cs:\boxplotvalue{upper whisker},1)
                      {\pgfmathprintnumber{\boxplotvalue{upper whisker}}}
                    ;
                    
            \addplot+[boxplot] 
                    table [y index=0] {Data/double/xfwhm.txt}
                    [above]
                    node at
                      (boxplot whisker cs:\boxplotvalue{lower whisker},1)
                      {\pgfmathprintnumber{\boxplotvalue{lower whisker}}}
                    node at
                      (boxplot box cs: \boxplotvalue{median},-1)
                      {\pgfmathprintnumber{\boxplotvalue{median}}}
                    node at
                      (boxplot whisker cs:\boxplotvalue{upper whisker},1)
                      {\pgfmathprintnumber{\boxplotvalue{upper whisker}}}
                    ;
            [
                       
            \end{axis}
        \end{tikzpicture}
    \caption{\textbf{Averaged Point Spread Analysis: Custom 3D Printed Mount, Shear Technique} Box plots of Average FWHM in examined agarose bead samples (m = 4).}
    \end{subfigure}
    ~
    \begin{subfigure}[t]{0.49\textwidth}
        \centering
            \begin{tikzpicture}
                 \begin{axis}
                 [
                scale = 1,
                xmin = 2,
                xmax = 15.5,
                ytick={1,2,3},
                yticklabels={Z',Y, X'},
                boxplot/box extend=0.25,
                ]
        
        \addplot+ [boxplot] 
                table [y index=0] {Data/double/nzfwhm.txt}
                [above]
                node at
                  (boxplot whisker cs:\boxplotvalue{lower whisker},1)
                  {\pgfmathprintnumber{\boxplotvalue{lower whisker}}}
                node at
                  (boxplot box cs: \boxplotvalue{median},-1)
                  {\pgfmathprintnumber{\boxplotvalue{median}}}
                node at
                  (boxplot whisker cs:\boxplotvalue{upper whisker},1)
                  {\pgfmathprintnumber{\boxplotvalue{upper whisker}}}
                ;
                
        \addplot+[boxplot] 
                table [y index=0] {Data/double/nyfwhm.txt}
                [above]
                node at
                  (boxplot box cs:\boxplotvalue{lower whisker},0.9)
                  {\pgfmathprintnumber{\boxplotvalue{lower whisker}}}
                node at
                  (boxplot box cs: \boxplotvalue{median},-1)
                  {\pgfmathprintnumber{\boxplotvalue{median}}}
                node at
                  (boxplot box cs:\boxplotvalue{upper whisker},1)
                  {\pgfmathprintnumber{\boxplotvalue{upper whisker}}}
                ;
                
        \addplot+[boxplot] 
                table [y index=0] {Data/double/nxfwhm.txt}
                [above]
                node at
                  (boxplot whisker cs:\boxplotvalue{lower whisker},1)
                  {\pgfmathprintnumber{\boxplotvalue{lower whisker}}}
                node at
                  (boxplot box cs: \boxplotvalue{median},-1)
                  {\pgfmathprintnumber{\boxplotvalue{median}}}
                node at
                  (boxplot whisker cs:\boxplotvalue{upper whisker},1)
                  {\pgfmathprintnumber{\boxplotvalue{upper whisker}}}
                ;
                [      
                \end{axis}
            \end{tikzpicture}
    \caption{\textbf{Averaged Point Spread Analysis: Custom 3D Printed Mount, No Shear Technique} Box plots of Average FWHM in examined agarose bead samples (m = 4).}
    \end{subfigure}

    \label{fig:enter-label}
    \caption{\textbf{Point Spread Analysis Comparison: Shear and No Shear Techniques.} 1 micron florescent beads in agarose solution examined in recordings: Shear Recording (n = 1182 beads), No Shear Recording (n = 5470 beads).}
\end{figure}

\begin{figure}[H]
    \centering
        \begin{subfigure}[t]{0.4\textwidth}
        \centering
        \begin{tikzpicture}
                \begin{axis}
                [
                scale = 0.85,
                xlabel = Raw Data File Size (GB),
                ylabel = Processed Data File Size (GB),
                xmin = 0,
                xmax = 18,
                ymax= 40,
                ymin = 0
                ]
                \addplot table {Data/Noshearfilesize.tex}
                ;
                \addplot table {Data/Shearfilesize.tex}
                        ; 
                \end{axis}
            \end{tikzpicture}
        \caption{\textbf{Change in file size of data after image processing}.}
        \label{fig:enter-label}
        \end{subfigure}
        \hspace{2em}
        ~
        \begin{subfigure}[t]{0.4\textwidth}
        \centering
         \begin{tikzpicture}
                \begin{axis}
                [
                scale = 0.85,
                xlabel = Input File Size (GB),
                ylabel = Processing Time (s),
                xmin = 0,
                xmax = 14,
                ymax= 300,
                ymin = 0
                ]
                \addplot table {Data/NoShearTime.tex}
                ;
                \addplot table {Data/ShearTime.tex}
                ; 
                \end{axis}
            \end{tikzpicture}
        \caption{\textbf{Change in affine transform processing time to input file size}.}
        \label{fig:enter-label}
        \end{subfigure}
    \caption{\textbf{Comparison of Data Processing Files Size and Processing Times}. Blue lines indicate trend for No Shear data, red lines indicates trend for Shear data.}
    \label{fig:enter-label}
\end{figure}

\begin{figure}[H]
\centering
    \begin{subfigure}[t]{0.49\textwidth}
        \centering
        \begin{tikzpicture}
            \begin{axis}
            [
            scale = 1,
            ytick={1,2,3},
            yticklabels={Z',Y, X'},
            boxplot/box extend=0.25,
            xmax = 25,
            xlabel = {Length ($\mu$m)}
            ]
    
   \addplot+[boxplot, mark=none] 
            table [y index=0] {Data/ROI1/zfwhm.txt}
            [above]
            node at
              (boxplot whisker cs:\boxplotvalue{lower whisker},1)
              {\pgfmathprintnumber{\boxplotvalue{lower whisker}}}
            node at
              (boxplot box cs: \boxplotvalue{median},-1)
              {\pgfmathprintnumber{\boxplotvalue{median}}}
            node at
              (boxplot whisker cs:\boxplotvalue{upper whisker},1)
              {\pgfmathprintnumber{\boxplotvalue{upper whisker}}}
            ;
            
    \addplot+[boxplot, mark=none] 
            table [y index=0] {Data/ROI1/yfwhm.txt}
            [above]
            node at
              (boxplot whisker cs:\boxplotvalue{lower whisker},1)
              {\pgfmathprintnumber{\boxplotvalue{lower whisker}}}
            node at
              (boxplot box cs: \boxplotvalue{median},1)
              {\pgfmathprintnumber{\boxplotvalue{median}}}
            node at
              (boxplot whisker cs:\boxplotvalue{upper whisker},1)
              {\pgfmathprintnumber{\boxplotvalue{upper whisker}}}
            ;
   \addplot+[boxplot, mark=none] 
            table [y index=0] {Data/ROI1/xfwhm.txt}
            [above]
            node at
              (boxplot whisker cs:\boxplotvalue{lower whisker},1)
              {\pgfmathprintnumber{\boxplotvalue{lower whisker}}}
            node at
              (boxplot box cs: \boxplotvalue{median},-1)
              {\pgfmathprintnumber{\boxplotvalue{median}}}
            node at
              (boxplot whisker cs:\boxplotvalue{upper whisker},1)
              {\pgfmathprintnumber{\boxplotvalue{upper whisker}}}
            ;
            \end{axis}
 
        \end{tikzpicture}
        \caption{\textbf{ROI 1 (n=43)}}
    \end{subfigure}
    \medskip
    ~
    \begin{subfigure}[t]{0.49\textwidth}
        \centering
        \begin{tikzpicture}
           \begin{axis}
            [
            scale = 1,
            ytick={1,2,3},
            yticklabels={Z',Y, X'},
            boxplot/box extend=0.25,
            xmax = 25,
            xlabel = {Length ($\mu$m)}
            ]
    
   \addplot+[boxplot, mark=none] 
            table [y index=0] {Data/ROI2/zfwhm.txt}
            [above]
            node at
              (boxplot whisker cs:\boxplotvalue{lower whisker},1)
              {\pgfmathprintnumber{\boxplotvalue{lower whisker}}}
            node at
              (boxplot box cs: \boxplotvalue{median},-1)
              {\pgfmathprintnumber{\boxplotvalue{median}}}
            node at
              (boxplot whisker cs:\boxplotvalue{upper whisker},1)
              {\pgfmathprintnumber{\boxplotvalue{upper whisker}}}
            ;
            
    \addplot+[boxplot, mark=none] 
            table [y index=0] {Data/ROI2/yfwhm.txt}
            [above]
            node at
              (boxplot whisker cs:\boxplotvalue{lower whisker},1)
              {\pgfmathprintnumber{\boxplotvalue{lower whisker}}}
            node at
              (boxplot box cs: \boxplotvalue{median},-1)
              {\pgfmathprintnumber{\boxplotvalue{median}}}
            node at
              (boxplot whisker cs:\boxplotvalue{upper whisker},1)
              {\pgfmathprintnumber{\boxplotvalue{upper whisker}}}
            ;
   \addplot+[boxplot, mark=none] 
            table [y index=0] {Data/ROI2/xfwhm.txt}
            [above]
            node at
              (boxplot whisker cs:\boxplotvalue{lower whisker},1)
              {\pgfmathprintnumber{\boxplotvalue{lower whisker}}}
            node at
              (boxplot box cs: \boxplotvalue{median},-1)
              {\pgfmathprintnumber{\boxplotvalue{median}}}
            node at
              (boxplot whisker cs:\boxplotvalue{upper whisker},1)
              {\pgfmathprintnumber{\boxplotvalue{upper whisker}}}
            ;
            
            \end{axis}
        \end{tikzpicture}
        \caption{\textbf{ROI 2 (n=133)}}
    \end{subfigure}
    \caption{\textbf{Distribution of cell nuclei dimensional measurements.} Median, upper and lower quartiles, max and min extrema shown. Measurements recorded in sample coordinate system (x', y, z').}
    \label{fig:enter-label}
\end{figure}

\begin{figure}[H]
        \centering
        \begin{tikzpicture}
            \begin{axis}
            [
            scale = 1,
            ytick={1,2,3},
            yticklabels={$Z'_{ROI_2}$,$Z'_{ROI_1}$,\empty},
            boxplot/box extend=0.25,
            xmin= 6,
            xmax = 18, 
            ymin = 0,
            ymax=3,
            xlabel = {Length ($\mu$m)}
            ]
    
     \addplot+[boxplot, mark=none] 
            table [y index=0] {Data/ROI2/zfwhm.txt}
            [above]
            node at
              (boxplot box cs: \boxplotvalue{median},1.1)
              {\pgfmathprintnumber{\boxplotvalue{median}}}
            ;
    \addplot+ [boxplot prepared={lower whisker=13.23, upper whisker=13.23,whisker extend = 4,draw position=1,every whisker/.style={magenta,ultra thick}}] 
                %table [y index=0] {Data/double/zfwhm.txt}
                coordinates {};
                    %;
    \addplot+ [boxplot prepared={lower whisker=7.76, upper whisker=7.76,whisker extend = 4,draw position=1,every whisker/.style={cyan,ultra thick}}]
                %table [y index=0] {Data/double/zfwhm.txt}
                coordinates {};
                    %;
    \addplot+[boxplot, mark=none] 
            table [y index=0] {Data/ROI1/zfwhm.txt}
            [above]
            node at
              (boxplot box cs: \boxplotvalue{median},1.1)
              {\pgfmathprintnumber{\boxplotvalue{median}}}
            ;
              \end{axis}
        \end{tikzpicture}
        \caption{\textbf{FWHM distribution of cell nuclei axial measurements in ROIs (Fig. 5.1(a-b)).} (N, ROI \#1 = 27 nuclei; N, ROI \#2 = 132 nuclei)  Median axial measurement, inner quartiles, minimum and maximum values are shown. magenta line = shear imaging method axial resolution limit (13.23 microns). Cyan line = no shear method axial resolution limit (7.76 microns).}
    \end{figure}