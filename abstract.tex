\chapter{Abstract}
Volumetric imaging is seen as the optimal means by which the three-dimensional structural remodelling in cardiac tissue post Myocardial Infarction (MI) can be observed for use in cardiovascular research.  This contrasts with the currently standard practice of two-dimensional histological analysis of cardiac tissue, which sees transmural sections of cardiac tissue under 400 microns in thickness to allow use of established light microscopy methods \cite{casero_transformation_2017}. This limits the ability of these traditional histology methods to provide sufficient information on the more complex, overall 3D alterations to the structure of the heart tissue post-MI.
 
Mesoscale light sheet fluorescence microscopy now provides a viable alternative to standard two-dimensional histology in cardiovascular research using advancements in tissue clearing, mounting, and data processing to allow mesoscale volumes of cardiac tissue ($cm^3$) to be imaged without the need for ultra fine sectioning \cite{voigt_mesospim_2019}. This imaging approach allows for high resolution, sectional imaging of individual planes of tissue to be performed with the tissue volume remaining intact. Recorded sections can then be recombined digitally to recreate the volumetric structure, allowing 3D alterations in tissue structure to be viewed without obstructions or artefacts created from manual tissue sectioning seen in 2D histology. 

This thesis aims to present an imaging pipeline that will allow cardiac tissue extracted from leporine animal models to be processed for use in imaging with the mesoSPIM, an open source, mesoscale light sheet microscopy system designed to allow for isometric resolution imaging. This is achieved through processing of tissue samples using established tissue clearing techniques to render optically transparent and by designing and optimizing novel mounting methodologies along with associated data acquisition and processing procedures. Using these pre and post image processing procedures, three-dimensional structural imaging of left ventricle samples in the mesoSPIM system was achieved. We address the viability of this imaging pipeline for use in cardiovascular research by utilizing the pipeline in three ongoing research projects examining: alterations in cardiac cellular innervation post-MI, differences in cardiac structure between surgical MI models, and the retention of spheroids injected into the sub-epicardium for use in regenerative medicine. This work serves to introduce this novel imaging pipeline to the wider cardiovascular research field as a potential new standard method by which three-dimensional alterations in cardiac tissue post-MI can be recorded analysed to further research and understanding of cardiovascular disease.
